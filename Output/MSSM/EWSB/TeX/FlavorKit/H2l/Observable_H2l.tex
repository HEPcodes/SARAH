\documentclass[A4,landscape]{article} 
\usepackage{amsmath}
\usepackage[T1]{fontenc}
\usepackage{amssymb}
\usepackage{feynmp}
\usepackage{hyperref}
\usepackage{longtable}
\DeclareGraphicsRule{*}{mps}{*}{}
\graphicspath{{./Diagrams/}}
\textwidth 25cm
\evensidemargin -1.0cm
\oddsidemargin -1.0cm
\begin{document}
\title{Analytical expressions for the form factors of H2l\\ in the MSSM } 
 \author{SARAH 4.5.8} 
 \maketitle 
 \vspace{10cm} 
This file was automatically generated by SARAH version 4.5.8.  \\ 
References: {\bf arXiv: 1309.7223 }, {\bf Comput.Phys.Commun.184:1792-1809,2011 (1207.0906) }, {\bf Comput.Phys.Commun.182:808-833,2011 (1002.0840) }, {\bf Comput.Phys.Commun.181:1077-1086,2010 (0909.2863) }, {\bf arXiv: 0806.0538 } \\ 
Package Homepage: projects.hepforge.org/sarah/ \\ 
by {\bf Florian Staub, fnstaub@th.physik.uni-bonn.de} 
 \pagebreak 
 \tableofcontents 
 \pagebreak 
\section{External states: ${e_{{i}}, \bar{e}_{{j}}, h_{{k}}}$} 
\subsection{Tree contributions, Propagator: $\text{Propagator}$} 

\begin{align} 
  OH2lSL= & -16 \Gamma^{\bar{e}e h ,L}_{j, i, k} Pi^2 \\ 
  OH2lSR= & -16 \Gamma^{\bar{e}e h ,R}_{j, i, k} Pi^2 \\ 
\end{align} 
\subsection{Wave contributions, Propagator: $\text{Propagator}$} 



 \begin{center}
\begin{fmffile}{Diagrams/H2lWaveNumberOfConsideredExternalStatesVWm1}
\fmfframe(20,20)(20,20){
\begin{fmfgraph*}(150,75)
\fmfleft{l2,l1}
\fmfright{r1,r2}
\fmf{plain}{v3,l2}
\fmf{dashes,label=$h_{{k}}$}{v3,v4}
\fmf{phantom}{v4,r1}
\fmf{phantom}{v4,r2}
\fmf{phantom}{l1,v3}
\fmffreeze
\fmf{plain}{l1,v1}
\fmf{dashes,right,tension=0.2,label=$A^0_{{a}}$}{v1,v2}
\fmf{plain,left,tension=0.2,label=$e_{{b}}$}{v1,v2}
\fmf{plain,label=$\bar{e}_{{c}}$}{v2,v3}
\fmflabel{$e_{{i}}$}{l1}
\fmflabel{$\bar{e}_{{j}}$}{l2}
\end{fmfgraph*}}
\end{fmffile}
\end{center}
 
\begin{align} 
I_1= & B_0(m^2_{e_{{i}}}, m^2_{e_{{b}}}, m^2_{A^0_{{a}}}) \\ 
I_2= & B_1(m^2_{e_{{i}}}, m^2_{e_{{b}}}, m^2_{A^0_{{a}}}) \\ 
  OH2lSL= & ( \Gamma^{\bar{e}e h ,L}_{j, c, k} (\Gamma^{\bar{e}e A^0 ,L}_{b, i, a} \Gamma^{\bar{e}e A^0 ,R}_{c, b, a} I_2 m^2_{e_{{i}}} - \Gamma^{\bar{e}e A^0 ,R}_{b, i, a} \Gamma^{\bar{e}e A^0 ,R}_{c, b, a} I_1 m_{e_{{i}}} m_{e_{{b}}} + \Gamma^{\bar{e}e A^0 ,R}_{b, i, a} \Gamma^{\bar{e}e A^0 ,L}_{c, b, a} I_2 m_{e_{{i}}} m_{e_{{c}}} - \Gamma^{\bar{e}e A^0 ,L}_{b, i, a} \Gamma^{\bar{e}e A^0 ,L}_{c, b, a} I_1 m_{e_{{b}}} m_{e_{{c}}}))/(m^2_{e_{{i}}} - m^2_{e_{{c}}}) \\ 
  OH2lSR= & ( \Gamma^{\bar{e}e h ,R}_{j, c, k} (\Gamma^{\bar{e}e A^0 ,R}_{b, i, a} \Gamma^{\bar{e}e A^0 ,L}_{c, b, a} I_2 m^2_{e_{{i}}} - \Gamma^{\bar{e}e A^0 ,L}_{b, i, a} \Gamma^{\bar{e}e A^0 ,L}_{c, b, a} I_1 m_{e_{{i}}} m_{e_{{b}}} + \Gamma^{\bar{e}e A^0 ,L}_{b, i, a} \Gamma^{\bar{e}e A^0 ,R}_{c, b, a} I_2 m_{e_{{i}}} m_{e_{{c}}} - \Gamma^{\bar{e}e A^0 ,R}_{b, i, a} \Gamma^{\bar{e}e A^0 ,R}_{c, b, a} I_1 m_{e_{{b}}} m_{e_{{c}}}))/(m^2_{e_{{i}}} - m^2_{e_{{c}}}) \\ 
\end{align} 


 \begin{center}
\begin{fmffile}{Diagrams/H2lWaveNumberOfConsideredExternalStatesVWm2}
\fmfframe(20,20)(20,20){
\begin{fmfgraph*}(150,75)
\fmfleft{l2,l1}
\fmfright{r1,r2}
\fmf{plain}{v3,l2}
\fmf{dashes,label=$h_{{k}}$}{v3,v4}
\fmf{phantom}{v4,r1}
\fmf{phantom}{v4,r2}
\fmf{phantom}{l1,v3}
\fmffreeze
\fmf{plain}{l1,v1}
\fmf{plain,right,tension=0.2,label=$\tilde{\chi}^0_{{a}}$}{v1,v2}
\fmf{dashes,left,tension=0.2,label=$\tilde{e}_{{b}}$}{v1,v2}
\fmf{plain,label=$\bar{e}_{{c}}$}{v2,v3}
\fmflabel{$e_{{i}}$}{l1}
\fmflabel{$\bar{e}_{{j}}$}{l2}
\end{fmfgraph*}}
\end{fmffile}
\end{center}
 
\begin{align} 
I_1= & B_0(m^2_{e_{{i}}}, m^2_{\tilde{\chi}^0_{{a}}}, m^2_{\tilde{e}_{{b}}}) \\ 
I_2= & B_1(m^2_{e_{{i}}}, m^2_{\tilde{\chi}^0_{{a}}}, m^2_{\tilde{e}_{{b}}}) \\ 
  OH2lSL= & ( \Gamma^{\bar{e}e h ,L}_{j, c, k} (\Gamma^{\tilde{\chi}^0 e \tilde{e}^*,L}_{a, i, b} \Gamma^{\bar{e}\tilde{\chi}^0 \tilde{e} ,R}_{c, a, b} I_2 m^2_{e_{{i}}} - \Gamma^{\tilde{\chi}^0 e \tilde{e}^*,R}_{a, i, b} \Gamma^{\bar{e}\tilde{\chi}^0 \tilde{e} ,R}_{c, a, b} I_1 m_{e_{{i}}} m_{\tilde{\chi}^0_{{a}}} + \Gamma^{\tilde{\chi}^0 e \tilde{e}^*,R}_{a, i, b} \Gamma^{\bar{e}\tilde{\chi}^0 \tilde{e} ,L}_{c, a, b} I_2 m_{e_{{i}}} m_{e_{{c}}} - \Gamma^{\tilde{\chi}^0 e \tilde{e}^*,L}_{a, i, b} \Gamma^{\bar{e}\tilde{\chi}^0 \tilde{e} ,L}_{c, a, b} I_1 m_{\tilde{\chi}^0_{{a}}} m_{e_{{c}}}))/(m^2_{e_{{i}}} - m^2_{e_{{c}}}) \\ 
  OH2lSR= & ( \Gamma^{\bar{e}e h ,R}_{j, c, k} (\Gamma^{\tilde{\chi}^0 e \tilde{e}^*,R}_{a, i, b} \Gamma^{\bar{e}\tilde{\chi}^0 \tilde{e} ,L}_{c, a, b} I_2 m^2_{e_{{i}}} - \Gamma^{\tilde{\chi}^0 e \tilde{e}^*,L}_{a, i, b} \Gamma^{\bar{e}\tilde{\chi}^0 \tilde{e} ,L}_{c, a, b} I_1 m_{e_{{i}}} m_{\tilde{\chi}^0_{{a}}} + \Gamma^{\tilde{\chi}^0 e \tilde{e}^*,L}_{a, i, b} \Gamma^{\bar{e}\tilde{\chi}^0 \tilde{e} ,R}_{c, a, b} I_2 m_{e_{{i}}} m_{e_{{c}}} - \Gamma^{\tilde{\chi}^0 e \tilde{e}^*,R}_{a, i, b} \Gamma^{\bar{e}\tilde{\chi}^0 \tilde{e} ,R}_{c, a, b} I_1 m_{\tilde{\chi}^0_{{a}}} m_{e_{{c}}}))/(m^2_{e_{{i}}} - m^2_{e_{{c}}}) \\ 
\end{align} 


 \begin{center}
\begin{fmffile}{Diagrams/H2lWaveNumberOfConsideredExternalStatesVWm3}
\fmfframe(20,20)(20,20){
\begin{fmfgraph*}(150,75)
\fmfleft{l2,l1}
\fmfright{r1,r2}
\fmf{plain}{v3,l2}
\fmf{dashes,label=$h_{{k}}$}{v3,v4}
\fmf{phantom}{v4,r1}
\fmf{phantom}{v4,r2}
\fmf{phantom}{l1,v3}
\fmffreeze
\fmf{plain}{l1,v1}
\fmf{dashes,right,tension=0.2,label=$h_{{a}}$}{v1,v2}
\fmf{plain,left,tension=0.2,label=$e_{{b}}$}{v1,v2}
\fmf{plain,label=$\bar{e}_{{c}}$}{v2,v3}
\fmflabel{$e_{{i}}$}{l1}
\fmflabel{$\bar{e}_{{j}}$}{l2}
\end{fmfgraph*}}
\end{fmffile}
\end{center}
 
\begin{align} 
I_1= & B_0(m^2_{e_{{i}}}, m^2_{e_{{b}}}, m^2_{h_{{a}}}) \\ 
I_2= & B_1(m^2_{e_{{i}}}, m^2_{e_{{b}}}, m^2_{h_{{a}}}) \\ 
  OH2lSL= & ( \Gamma^{\bar{e}e h ,L}_{j, c, k} (\Gamma^{\bar{e}e h ,L}_{b, i, a} \Gamma^{\bar{e}e h ,R}_{c, b, a} I_2 m^2_{e_{{i}}} - \Gamma^{\bar{e}e h ,R}_{b, i, a} \Gamma^{\bar{e}e h ,R}_{c, b, a} I_1 m_{e_{{i}}} m_{e_{{b}}} + \Gamma^{\bar{e}e h ,R}_{b, i, a} \Gamma^{\bar{e}e h ,L}_{c, b, a} I_2 m_{e_{{i}}} m_{e_{{c}}} - \Gamma^{\bar{e}e h ,L}_{b, i, a} \Gamma^{\bar{e}e h ,L}_{c, b, a} I_1 m_{e_{{b}}} m_{e_{{c}}}))/(m^2_{e_{{i}}} - m^2_{e_{{c}}}) \\ 
  OH2lSR= & ( \Gamma^{\bar{e}e h ,R}_{j, c, k} (\Gamma^{\bar{e}e h ,R}_{b, i, a} \Gamma^{\bar{e}e h ,L}_{c, b, a} I_2 m^2_{e_{{i}}} - \Gamma^{\bar{e}e h ,L}_{b, i, a} \Gamma^{\bar{e}e h ,L}_{c, b, a} I_1 m_{e_{{i}}} m_{e_{{b}}} + \Gamma^{\bar{e}e h ,L}_{b, i, a} \Gamma^{\bar{e}e h ,R}_{c, b, a} I_2 m_{e_{{i}}} m_{e_{{c}}} - \Gamma^{\bar{e}e h ,R}_{b, i, a} \Gamma^{\bar{e}e h ,R}_{c, b, a} I_1 m_{e_{{b}}} m_{e_{{c}}}))/(m^2_{e_{{i}}} - m^2_{e_{{c}}}) \\ 
\end{align} 


 \begin{center}
\begin{fmffile}{Diagrams/H2lWaveNumberOfConsideredExternalStatesVWm4}
\fmfframe(20,20)(20,20){
\begin{fmfgraph*}(150,75)
\fmfleft{l2,l1}
\fmfright{r1,r2}
\fmf{plain}{v3,l2}
\fmf{dashes,label=$h_{{k}}$}{v3,v4}
\fmf{phantom}{v4,r1}
\fmf{phantom}{v4,r2}
\fmf{phantom}{l1,v3}
\fmffreeze
\fmf{plain}{l1,v1}
\fmf{wiggly,right,tension=0.2,label=$\gamma$}{v1,v2}
\fmf{plain,left,tension=0.2,label=$e_{{b}}$}{v1,v2}
\fmf{plain,label=$\bar{e}_{{c}}$}{v2,v3}
\fmflabel{$e_{{i}}$}{l1}
\fmflabel{$\bar{e}_{{j}}$}{l2}
\end{fmfgraph*}}
\end{fmffile}
\end{center}
 
\begin{align} 
I_1= & B_0(m^2_{e_{{i}}}, m^2_{e_{{b}}}, m^2_{\gamma}) \\ 
I_2= & B_1(m^2_{e_{{i}}}, m^2_{e_{{b}}}, m^2_{\gamma}) \\ 
  OH2lSL= & ( \Gamma^{\bar{e}e h ,L}_{j, c, k} (\Gamma^{\bar{e}e \gamma ,R}_{b, i} m_{e_{{i}}} (-2 \Gamma^{\bar{e}e \gamma ,L}_{c, b} (1 - 2 I_1) m_{e_{{b}}} + \Gamma^{\bar{e}e \gamma ,R}_{c, b} (1 + 2 I_2) m_{e_{{c}}}) + \Gamma^{\bar{e}e \gamma ,L}_{b, i} (\Gamma^{\bar{e}e \gamma ,L}_{c, b} (1 + 2 I_2) m^2_{e_{{i}}} - 2 \Gamma^{\bar{e}e \gamma ,R}_{c, b} (1 - 2 I_1) m_{e_{{b}}} m_{e_{{c}}})))/(m^2_{e_{{i}}} - m^2_{e_{{c}}}) \\ 
  OH2lSR= & ( \Gamma^{\bar{e}e h ,R}_{j, c, k} (\Gamma^{\bar{e}e \gamma ,L}_{b, i} m_{e_{{i}}} (-2 \Gamma^{\bar{e}e \gamma ,R}_{c, b} (1 - 2 I_1) m_{e_{{b}}} + \Gamma^{\bar{e}e \gamma ,L}_{c, b} (1 + 2 I_2) m_{e_{{c}}}) + \Gamma^{\bar{e}e \gamma ,R}_{b, i} (\Gamma^{\bar{e}e \gamma ,R}_{c, b} (1 + 2 I_2) m^2_{e_{{i}}} - 2 \Gamma^{\bar{e}e \gamma ,L}_{c, b} (1 - 2 I_1) m_{e_{{b}}} m_{e_{{c}}})))/(m^2_{e_{{i}}} - m^2_{e_{{c}}}) \\ 
\end{align} 


 \begin{center}
\begin{fmffile}{Diagrams/H2lWaveNumberOfConsideredExternalStatesVWm5}
\fmfframe(20,20)(20,20){
\begin{fmfgraph*}(150,75)
\fmfleft{l2,l1}
\fmfright{r1,r2}
\fmf{plain}{v3,l2}
\fmf{dashes,label=$h_{{k}}$}{v3,v4}
\fmf{phantom}{v4,r1}
\fmf{phantom}{v4,r2}
\fmf{phantom}{l1,v3}
\fmffreeze
\fmf{plain}{l1,v1}
\fmf{wiggly,right,tension=0.2,label=$Z$}{v1,v2}
\fmf{plain,left,tension=0.2,label=$e_{{b}}$}{v1,v2}
\fmf{plain,label=$\bar{e}_{{c}}$}{v2,v3}
\fmflabel{$e_{{i}}$}{l1}
\fmflabel{$\bar{e}_{{j}}$}{l2}
\end{fmfgraph*}}
\end{fmffile}
\end{center}
 
\begin{align} 
I_1= & B_0(m^2_{e_{{i}}}, m^2_{e_{{b}}}, m^2_{Z}) \\ 
I_2= & B_1(m^2_{e_{{i}}}, m^2_{e_{{b}}}, m^2_{Z}) \\ 
  OH2lSL= & ( \Gamma^{\bar{e}e h ,L}_{j, c, k} (\Gamma^{\bar{e}e Z ,R}_{b, i} m_{e_{{i}}} (-2 \Gamma^{\bar{e}e Z ,L}_{c, b} (1 - 2 I_1) m_{e_{{b}}} + \Gamma^{\bar{e}e Z ,R}_{c, b} (1 + 2 I_2) m_{e_{{c}}}) + \Gamma^{\bar{e}e Z ,L}_{b, i} (\Gamma^{\bar{e}e Z ,L}_{c, b} (1 + 2 I_2) m^2_{e_{{i}}} - 2 \Gamma^{\bar{e}e Z ,R}_{c, b} (1 - 2 I_1) m_{e_{{b}}} m_{e_{{c}}})))/(m^2_{e_{{i}}} - m^2_{e_{{c}}}) \\ 
  OH2lSR= & ( \Gamma^{\bar{e}e h ,R}_{j, c, k} (\Gamma^{\bar{e}e Z ,L}_{b, i} m_{e_{{i}}} (-2 \Gamma^{\bar{e}e Z ,R}_{c, b} (1 - 2 I_1) m_{e_{{b}}} + \Gamma^{\bar{e}e Z ,L}_{c, b} (1 + 2 I_2) m_{e_{{c}}}) + \Gamma^{\bar{e}e Z ,R}_{b, i} (\Gamma^{\bar{e}e Z ,R}_{c, b} (1 + 2 I_2) m^2_{e_{{i}}} - 2 \Gamma^{\bar{e}e Z ,L}_{c, b} (1 - 2 I_1) m_{e_{{b}}} m_{e_{{c}}})))/(m^2_{e_{{i}}} - m^2_{e_{{c}}}) \\ 
\end{align} 


 \begin{center}
\begin{fmffile}{Diagrams/H2lWaveNumberOfConsideredExternalStatesVWm6}
\fmfframe(20,20)(20,20){
\begin{fmfgraph*}(150,75)
\fmfleft{l2,l1}
\fmfright{r1,r2}
\fmf{plain}{v3,l2}
\fmf{dashes,label=$h_{{k}}$}{v3,v4}
\fmf{phantom}{v4,r1}
\fmf{phantom}{v4,r2}
\fmf{phantom}{l1,v3}
\fmffreeze
\fmf{plain}{l1,v1}
\fmf{plain,right,tension=0.2,label=$\tilde{\chi}^+_{{a}}$}{v1,v2}
\fmf{dashes,left,tension=0.2,label=$\tilde{\nu}_{{b}}$}{v1,v2}
\fmf{plain,label=$\bar{e}_{{c}}$}{v2,v3}
\fmflabel{$e_{{i}}$}{l1}
\fmflabel{$\bar{e}_{{j}}$}{l2}
\end{fmfgraph*}}
\end{fmffile}
\end{center}
 
\begin{align} 
I_1= & B_0(m^2_{e_{{i}}}, m^2_{\tilde{\chi}^-_{{a}}}, m^2_{\tilde{\nu}_{{b}}}) \\ 
I_2= & B_1(m^2_{e_{{i}}}, m^2_{\tilde{\chi}^-_{{a}}}, m^2_{\tilde{\nu}_{{b}}}) \\ 
  OH2lSL= & ( \Gamma^{\bar{e}e h ,L}_{j, c, k} (\Gamma^{\tilde{\chi}^+e \tilde{\nu}^*,L}_{a, i, b} \Gamma^{\bar{e}\tilde{\chi}^- \tilde{\nu} ,R}_{c, a, b} I_2 m^2_{e_{{i}}} - \Gamma^{\tilde{\chi}^+e \tilde{\nu}^*,R}_{a, i, b} \Gamma^{\bar{e}\tilde{\chi}^- \tilde{\nu} ,R}_{c, a, b} I_1 m_{e_{{i}}} m_{\tilde{\chi}^-_{{a}}} + \Gamma^{\tilde{\chi}^+e \tilde{\nu}^*,R}_{a, i, b} \Gamma^{\bar{e}\tilde{\chi}^- \tilde{\nu} ,L}_{c, a, b} I_2 m_{e_{{i}}} m_{e_{{c}}} - \Gamma^{\tilde{\chi}^+e \tilde{\nu}^*,L}_{a, i, b} \Gamma^{\bar{e}\tilde{\chi}^- \tilde{\nu} ,L}_{c, a, b} I_1 m_{\tilde{\chi}^-_{{a}}} m_{e_{{c}}}))/(m^2_{e_{{i}}} - m^2_{e_{{c}}}) \\ 
  OH2lSR= & ( \Gamma^{\bar{e}e h ,R}_{j, c, k} (\Gamma^{\tilde{\chi}^+e \tilde{\nu}^*,R}_{a, i, b} \Gamma^{\bar{e}\tilde{\chi}^- \tilde{\nu} ,L}_{c, a, b} I_2 m^2_{e_{{i}}} - \Gamma^{\tilde{\chi}^+e \tilde{\nu}^*,L}_{a, i, b} \Gamma^{\bar{e}\tilde{\chi}^- \tilde{\nu} ,L}_{c, a, b} I_1 m_{e_{{i}}} m_{\tilde{\chi}^-_{{a}}} + \Gamma^{\tilde{\chi}^+e \tilde{\nu}^*,L}_{a, i, b} \Gamma^{\bar{e}\tilde{\chi}^- \tilde{\nu} ,R}_{c, a, b} I_2 m_{e_{{i}}} m_{e_{{c}}} - \Gamma^{\tilde{\chi}^+e \tilde{\nu}^*,R}_{a, i, b} \Gamma^{\bar{e}\tilde{\chi}^- \tilde{\nu} ,R}_{c, a, b} I_1 m_{\tilde{\chi}^-_{{a}}} m_{e_{{c}}}))/(m^2_{e_{{i}}} - m^2_{e_{{c}}}) \\ 
\end{align} 


 \begin{center}
\begin{fmffile}{Diagrams/H2lWaveNumberOfConsideredExternalStatesVWm7}
\fmfframe(20,20)(20,20){
\begin{fmfgraph*}(150,75)
\fmfleft{l2,l1}
\fmfright{r1,r2}
\fmf{plain}{v3,l2}
\fmf{dashes,label=$h_{{k}}$}{v3,v4}
\fmf{phantom}{v4,r1}
\fmf{phantom}{v4,r2}
\fmf{phantom}{l1,v3}
\fmffreeze
\fmf{plain}{l1,v1}
\fmf{plain,right,tension=0.2,label=$\bar{\nu}_{{a}}$}{v1,v2}
\fmf{dashes,left,tension=0.2,label=$H^-_{{b}}$}{v1,v2}
\fmf{plain,label=$\bar{e}_{{c}}$}{v2,v3}
\fmflabel{$e_{{i}}$}{l1}
\fmflabel{$\bar{e}_{{j}}$}{l2}
\end{fmfgraph*}}
\end{fmffile}
\end{center}
 
\begin{align} 
I_1= & B_0(m^2_{e_{{i}}}, m^2_{\nu_{{a}}}, m^2_{H^-_{{b}}}) \\ 
I_2= & B_1(m^2_{e_{{i}}}, m^2_{\nu_{{a}}}, m^2_{H^-_{{b}}}) \\ 
  OH2lSL= & ( \Gamma^{\bar{e}e h ,L}_{j, c, k} (\Gamma^{\bar{\nu}e H^+,L}_{a, i, b} \Gamma^{\bar{e}\nu H^- ,R}_{c, a, b} I_2 m^2_{e_{{i}}} - \Gamma^{\bar{\nu}e H^+,R}_{a, i, b} \Gamma^{\bar{e}\nu H^- ,R}_{c, a, b} I_1 m_{e_{{i}}} m_{\nu_{{a}}} + \Gamma^{\bar{\nu}e H^+,R}_{a, i, b} \Gamma^{\bar{e}\nu H^- ,L}_{c, a, b} I_2 m_{e_{{i}}} m_{e_{{c}}} - \Gamma^{\bar{\nu}e H^+,L}_{a, i, b} \Gamma^{\bar{e}\nu H^- ,L}_{c, a, b} I_1 m_{\nu_{{a}}} m_{e_{{c}}}))/(m^2_{e_{{i}}} - m^2_{e_{{c}}}) \\ 
  OH2lSR= & ( \Gamma^{\bar{e}e h ,R}_{j, c, k} (\Gamma^{\bar{\nu}e H^+,R}_{a, i, b} \Gamma^{\bar{e}\nu H^- ,L}_{c, a, b} I_2 m^2_{e_{{i}}} - \Gamma^{\bar{\nu}e H^+,L}_{a, i, b} \Gamma^{\bar{e}\nu H^- ,L}_{c, a, b} I_1 m_{e_{{i}}} m_{\nu_{{a}}} + \Gamma^{\bar{\nu}e H^+,L}_{a, i, b} \Gamma^{\bar{e}\nu H^- ,R}_{c, a, b} I_2 m_{e_{{i}}} m_{e_{{c}}} - \Gamma^{\bar{\nu}e H^+,R}_{a, i, b} \Gamma^{\bar{e}\nu H^- ,R}_{c, a, b} I_1 m_{\nu_{{a}}} m_{e_{{c}}}))/(m^2_{e_{{i}}} - m^2_{e_{{c}}}) \\ 
\end{align} 


 \begin{center}
\begin{fmffile}{Diagrams/H2lWaveNumberOfConsideredExternalStatesVWm8}
\fmfframe(20,20)(20,20){
\begin{fmfgraph*}(150,75)
\fmfleft{l2,l1}
\fmfright{r1,r2}
\fmf{plain}{v3,l2}
\fmf{dashes,label=$h_{{k}}$}{v3,v4}
\fmf{phantom}{v4,r1}
\fmf{phantom}{v4,r2}
\fmf{phantom}{l1,v3}
\fmffreeze
\fmf{plain}{l1,v1}
\fmf{plain,right,tension=0.2,label=$\bar{\nu}_{{a}}$}{v1,v2}
\fmf{wiggly,left,tension=0.2,label=$W^-$}{v1,v2}
\fmf{plain,label=$\bar{e}_{{c}}$}{v2,v3}
\fmflabel{$e_{{i}}$}{l1}
\fmflabel{$\bar{e}_{{j}}$}{l2}
\end{fmfgraph*}}
\end{fmffile}
\end{center}
 
\begin{align} 
I_1= & B_0(m^2_{e_{{i}}}, m^2_{\nu_{{a}}}, m^2_{W^-}) \\ 
I_2= & B_1(m^2_{e_{{i}}}, m^2_{\nu_{{a}}}, m^2_{W^-}) \\ 
  OH2lSL= & ( \Gamma^{\bar{e}e h ,L}_{j, c, k} (\Gamma^{\bar{\nu}e W^+,R}_{a, i} m_{e_{{i}}} (-2 \Gamma^{\bar{e}\nu W^- ,L}_{c, a} (1 - 2 I_1) m_{\nu_{{a}}} + \Gamma^{\bar{e}\nu W^- ,R}_{c, a} (1 + 2 I_2) m_{e_{{c}}}) + \Gamma^{\bar{\nu}e W^+,L}_{a, i} (\Gamma^{\bar{e}\nu W^- ,L}_{c, a} (1 + 2 I_2) m^2_{e_{{i}}} - 2 \Gamma^{\bar{e}\nu W^- ,R}_{c, a} (1 - 2 I_1) m_{\nu_{{a}}} m_{e_{{c}}})))/(m^2_{e_{{i}}} - m^2_{e_{{c}}}) \\ 
  OH2lSR= & ( \Gamma^{\bar{e}e h ,R}_{j, c, k} (\Gamma^{\bar{\nu}e W^+,L}_{a, i} m_{e_{{i}}} (-2 \Gamma^{\bar{e}\nu W^- ,R}_{c, a} (1 - 2 I_1) m_{\nu_{{a}}} + \Gamma^{\bar{e}\nu W^- ,L}_{c, a} (1 + 2 I_2) m_{e_{{c}}}) + \Gamma^{\bar{\nu}e W^+,R}_{a, i} (\Gamma^{\bar{e}\nu W^- ,R}_{c, a} (1 + 2 I_2) m^2_{e_{{i}}} - 2 \Gamma^{\bar{e}\nu W^- ,L}_{c, a} (1 - 2 I_1) m_{\nu_{{a}}} m_{e_{{c}}})))/(m^2_{e_{{i}}} - m^2_{e_{{c}}}) \\ 
\end{align} 


 \begin{center}
\begin{fmffile}{Diagrams/H2lWaveNumberOfConsideredExternalStatesVWm9}
\fmfframe(20,20)(20,20){
\begin{fmfgraph*}(150,75)
\fmfleft{l1,l2}
\fmfright{r1,r2}
\fmf{plain}{v3,l2}
\fmf{dashes,label=$h_{{k}}$}{v3,v4}
\fmf{phantom}{v4,r1}
\fmf{phantom}{v4,r2}
\fmf{phantom}{l1,v3}
\fmffreeze
\fmf{plain}{l1,v1}
\fmf{plain,right,tension=0.2,label=$e_{{a}}$}{v1,v2}
\fmf{dashes,left,tension=0.2,label=$A^0_{{b}}$}{v1,v2}
\fmf{plain,label=$e_{{c}}$}{v2,v3}
\fmflabel{$e_{{i}}$}{l2}
\fmflabel{$\bar{e}_{{j}}$}{l1}
\end{fmfgraph*}}
\end{fmffile}
\end{center}
 
\begin{align} 
I_1= & B_0(m^2_{e_{{j}}}, m^2_{e_{{a}}}, m^2_{A^0_{{b}}}) \\ 
I_2= & B_1(m^2_{e_{{j}}}, m^2_{e_{{a}}}, m^2_{A^0_{{b}}}) \\ 
  OH2lSL= & ( \Gamma^{\bar{e}e h ,L}_{c, i, k} (\Gamma^{\bar{e}e A^0 ,L}_{j, a, b} \Gamma^{\bar{e}e A^0 ,R}_{a, c, b} I_2 m^2_{e_{{j}}} - \Gamma^{\bar{e}e A^0 ,R}_{j, a, b} \Gamma^{\bar{e}e A^0 ,R}_{a, c, b} I_1 m_{e_{{j}}} m_{e_{{a}}} + \Gamma^{\bar{e}e A^0 ,R}_{j, a, b} \Gamma^{\bar{e}e A^0 ,L}_{a, c, b} I_2 m_{e_{{j}}} m_{e_{{c}}} - \Gamma^{\bar{e}e A^0 ,L}_{j, a, b} \Gamma^{\bar{e}e A^0 ,L}_{a, c, b} I_1 m_{e_{{a}}} m_{e_{{c}}}))/(m^2_{e_{{j}}} - m^2_{e_{{c}}}) \\ 
  OH2lSR= & ( \Gamma^{\bar{e}e h ,R}_{c, i, k} (\Gamma^{\bar{e}e A^0 ,R}_{j, a, b} \Gamma^{\bar{e}e A^0 ,L}_{a, c, b} I_2 m^2_{e_{{j}}} - \Gamma^{\bar{e}e A^0 ,L}_{j, a, b} \Gamma^{\bar{e}e A^0 ,L}_{a, c, b} I_1 m_{e_{{j}}} m_{e_{{a}}} + \Gamma^{\bar{e}e A^0 ,L}_{j, a, b} \Gamma^{\bar{e}e A^0 ,R}_{a, c, b} I_2 m_{e_{{j}}} m_{e_{{c}}} - \Gamma^{\bar{e}e A^0 ,R}_{j, a, b} \Gamma^{\bar{e}e A^0 ,R}_{a, c, b} I_1 m_{e_{{a}}} m_{e_{{c}}}))/(m^2_{e_{{j}}} - m^2_{e_{{c}}}) \\ 
\end{align} 


 \begin{center}
\begin{fmffile}{Diagrams/H2lWaveNumberOfConsideredExternalStatesVWm10}
\fmfframe(20,20)(20,20){
\begin{fmfgraph*}(150,75)
\fmfleft{l1,l2}
\fmfright{r1,r2}
\fmf{plain}{v3,l2}
\fmf{dashes,label=$h_{{k}}$}{v3,v4}
\fmf{phantom}{v4,r1}
\fmf{phantom}{v4,r2}
\fmf{phantom}{l1,v3}
\fmffreeze
\fmf{plain}{l1,v1}
\fmf{dashes,right,tension=0.2,label=$\tilde{e}_{{a}}$}{v1,v2}
\fmf{plain,left,tension=0.2,label=$\tilde{\chi}^0_{{b}}$}{v1,v2}
\fmf{plain,label=$e_{{c}}$}{v2,v3}
\fmflabel{$e_{{i}}$}{l2}
\fmflabel{$\bar{e}_{{j}}$}{l1}
\end{fmfgraph*}}
\end{fmffile}
\end{center}
 
\begin{align} 
I_1= & B_0(m^2_{e_{{j}}}, m^2_{\tilde{\chi}^0_{{b}}}, m^2_{\tilde{e}_{{a}}}) \\ 
I_2= & B_1(m^2_{e_{{j}}}, m^2_{\tilde{\chi}^0_{{b}}}, m^2_{\tilde{e}_{{a}}}) \\ 
  OH2lSL= & ( \Gamma^{\bar{e}e h ,L}_{c, i, k} (\Gamma^{\bar{e}\tilde{\chi}^0 \tilde{e} ,L}_{j, b, a} \Gamma^{\tilde{\chi}^0 e \tilde{e}^*,R}_{b, c, a} I_2 m^2_{e_{{j}}} - \Gamma^{\bar{e}\tilde{\chi}^0 \tilde{e} ,R}_{j, b, a} \Gamma^{\tilde{\chi}^0 e \tilde{e}^*,R}_{b, c, a} I_1 m_{e_{{j}}} m_{\tilde{\chi}^0_{{b}}} + \Gamma^{\bar{e}\tilde{\chi}^0 \tilde{e} ,R}_{j, b, a} \Gamma^{\tilde{\chi}^0 e \tilde{e}^*,L}_{b, c, a} I_2 m_{e_{{j}}} m_{e_{{c}}} - \Gamma^{\bar{e}\tilde{\chi}^0 \tilde{e} ,L}_{j, b, a} \Gamma^{\tilde{\chi}^0 e \tilde{e}^*,L}_{b, c, a} I_1 m_{\tilde{\chi}^0_{{b}}} m_{e_{{c}}}))/(m^2_{e_{{j}}} - m^2_{e_{{c}}}) \\ 
  OH2lSR= & ( \Gamma^{\bar{e}e h ,R}_{c, i, k} (\Gamma^{\bar{e}\tilde{\chi}^0 \tilde{e} ,R}_{j, b, a} \Gamma^{\tilde{\chi}^0 e \tilde{e}^*,L}_{b, c, a} I_2 m^2_{e_{{j}}} - \Gamma^{\bar{e}\tilde{\chi}^0 \tilde{e} ,L}_{j, b, a} \Gamma^{\tilde{\chi}^0 e \tilde{e}^*,L}_{b, c, a} I_1 m_{e_{{j}}} m_{\tilde{\chi}^0_{{b}}} + \Gamma^{\bar{e}\tilde{\chi}^0 \tilde{e} ,L}_{j, b, a} \Gamma^{\tilde{\chi}^0 e \tilde{e}^*,R}_{b, c, a} I_2 m_{e_{{j}}} m_{e_{{c}}} - \Gamma^{\bar{e}\tilde{\chi}^0 \tilde{e} ,R}_{j, b, a} \Gamma^{\tilde{\chi}^0 e \tilde{e}^*,R}_{b, c, a} I_1 m_{\tilde{\chi}^0_{{b}}} m_{e_{{c}}}))/(m^2_{e_{{j}}} - m^2_{e_{{c}}}) \\ 
\end{align} 


 \begin{center}
\begin{fmffile}{Diagrams/H2lWaveNumberOfConsideredExternalStatesVWm11}
\fmfframe(20,20)(20,20){
\begin{fmfgraph*}(150,75)
\fmfleft{l1,l2}
\fmfright{r1,r2}
\fmf{plain}{v3,l2}
\fmf{dashes,label=$h_{{k}}$}{v3,v4}
\fmf{phantom}{v4,r1}
\fmf{phantom}{v4,r2}
\fmf{phantom}{l1,v3}
\fmffreeze
\fmf{plain}{l1,v1}
\fmf{plain,right,tension=0.2,label=$e_{{a}}$}{v1,v2}
\fmf{dashes,left,tension=0.2,label=$h_{{b}}$}{v1,v2}
\fmf{plain,label=$e_{{c}}$}{v2,v3}
\fmflabel{$e_{{i}}$}{l2}
\fmflabel{$\bar{e}_{{j}}$}{l1}
\end{fmfgraph*}}
\end{fmffile}
\end{center}
 
\begin{align} 
I_1= & B_0(m^2_{e_{{j}}}, m^2_{e_{{a}}}, m^2_{h_{{b}}}) \\ 
I_2= & B_1(m^2_{e_{{j}}}, m^2_{e_{{a}}}, m^2_{h_{{b}}}) \\ 
  OH2lSL= & ( \Gamma^{\bar{e}e h ,L}_{c, i, k} (\Gamma^{\bar{e}e h ,L}_{j, a, b} \Gamma^{\bar{e}e h ,R}_{a, c, b} I_2 m^2_{e_{{j}}} - \Gamma^{\bar{e}e h ,R}_{j, a, b} \Gamma^{\bar{e}e h ,R}_{a, c, b} I_1 m_{e_{{j}}} m_{e_{{a}}} + \Gamma^{\bar{e}e h ,R}_{j, a, b} \Gamma^{\bar{e}e h ,L}_{a, c, b} I_2 m_{e_{{j}}} m_{e_{{c}}} - \Gamma^{\bar{e}e h ,L}_{j, a, b} \Gamma^{\bar{e}e h ,L}_{a, c, b} I_1 m_{e_{{a}}} m_{e_{{c}}}))/(m^2_{e_{{j}}} - m^2_{e_{{c}}}) \\ 
  OH2lSR= & ( \Gamma^{\bar{e}e h ,R}_{c, i, k} (\Gamma^{\bar{e}e h ,R}_{j, a, b} \Gamma^{\bar{e}e h ,L}_{a, c, b} I_2 m^2_{e_{{j}}} - \Gamma^{\bar{e}e h ,L}_{j, a, b} \Gamma^{\bar{e}e h ,L}_{a, c, b} I_1 m_{e_{{j}}} m_{e_{{a}}} + \Gamma^{\bar{e}e h ,L}_{j, a, b} \Gamma^{\bar{e}e h ,R}_{a, c, b} I_2 m_{e_{{j}}} m_{e_{{c}}} - \Gamma^{\bar{e}e h ,R}_{j, a, b} \Gamma^{\bar{e}e h ,R}_{a, c, b} I_1 m_{e_{{a}}} m_{e_{{c}}}))/(m^2_{e_{{j}}} - m^2_{e_{{c}}}) \\ 
\end{align} 


 \begin{center}
\begin{fmffile}{Diagrams/H2lWaveNumberOfConsideredExternalStatesVWm12}
\fmfframe(20,20)(20,20){
\begin{fmfgraph*}(150,75)
\fmfleft{l1,l2}
\fmfright{r1,r2}
\fmf{plain}{v3,l2}
\fmf{dashes,label=$h_{{k}}$}{v3,v4}
\fmf{phantom}{v4,r1}
\fmf{phantom}{v4,r2}
\fmf{phantom}{l1,v3}
\fmffreeze
\fmf{plain}{l1,v1}
\fmf{plain,right,tension=0.2,label=$e_{{a}}$}{v1,v2}
\fmf{wiggly,left,tension=0.2,label=$\gamma$}{v1,v2}
\fmf{plain,label=$e_{{c}}$}{v2,v3}
\fmflabel{$e_{{i}}$}{l2}
\fmflabel{$\bar{e}_{{j}}$}{l1}
\end{fmfgraph*}}
\end{fmffile}
\end{center}
 
\begin{align} 
I_1= & B_0(m^2_{e_{{j}}}, m^2_{e_{{a}}}, m^2_{\gamma}) \\ 
I_2= & B_1(m^2_{e_{{j}}}, m^2_{e_{{a}}}, m^2_{\gamma}) \\ 
  OH2lSL= & ( \Gamma^{\bar{e}e h ,L}_{c, i, k} (\Gamma^{\bar{e}e \gamma ,L}_{j, a} m_{e_{{j}}} (-2 \Gamma^{\bar{e}e \gamma ,R}_{a, c} (1 - 2 I_1) m_{e_{{a}}} + \Gamma^{\bar{e}e \gamma ,L}_{a, c} (1 + 2 I_2) m_{e_{{c}}}) + \Gamma^{\bar{e}e \gamma ,R}_{j, a} (\Gamma^{\bar{e}e \gamma ,R}_{a, c} (1 + 2 I_2) m^2_{e_{{j}}} - 2 \Gamma^{\bar{e}e \gamma ,L}_{a, c} (1 - 2 I_1) m_{e_{{a}}} m_{e_{{c}}})))/(m^2_{e_{{j}}} - m^2_{e_{{c}}}) \\ 
  OH2lSR= & ( \Gamma^{\bar{e}e h ,R}_{c, i, k} (\Gamma^{\bar{e}e \gamma ,R}_{j, a} m_{e_{{j}}} (-2 \Gamma^{\bar{e}e \gamma ,L}_{a, c} (1 - 2 I_1) m_{e_{{a}}} + \Gamma^{\bar{e}e \gamma ,R}_{a, c} (1 + 2 I_2) m_{e_{{c}}}) + \Gamma^{\bar{e}e \gamma ,L}_{j, a} (\Gamma^{\bar{e}e \gamma ,L}_{a, c} (1 + 2 I_2) m^2_{e_{{j}}} - 2 \Gamma^{\bar{e}e \gamma ,R}_{a, c} (1 - 2 I_1) m_{e_{{a}}} m_{e_{{c}}})))/(m^2_{e_{{j}}} - m^2_{e_{{c}}}) \\ 
\end{align} 


 \begin{center}
\begin{fmffile}{Diagrams/H2lWaveNumberOfConsideredExternalStatesVWm13}
\fmfframe(20,20)(20,20){
\begin{fmfgraph*}(150,75)
\fmfleft{l1,l2}
\fmfright{r1,r2}
\fmf{plain}{v3,l2}
\fmf{dashes,label=$h_{{k}}$}{v3,v4}
\fmf{phantom}{v4,r1}
\fmf{phantom}{v4,r2}
\fmf{phantom}{l1,v3}
\fmffreeze
\fmf{plain}{l1,v1}
\fmf{plain,right,tension=0.2,label=$e_{{a}}$}{v1,v2}
\fmf{wiggly,left,tension=0.2,label=$Z$}{v1,v2}
\fmf{plain,label=$e_{{c}}$}{v2,v3}
\fmflabel{$e_{{i}}$}{l2}
\fmflabel{$\bar{e}_{{j}}$}{l1}
\end{fmfgraph*}}
\end{fmffile}
\end{center}
 
\begin{align} 
I_1= & B_0(m^2_{e_{{j}}}, m^2_{e_{{a}}}, m^2_{Z}) \\ 
I_2= & B_1(m^2_{e_{{j}}}, m^2_{e_{{a}}}, m^2_{Z}) \\ 
  OH2lSL= & ( \Gamma^{\bar{e}e h ,L}_{c, i, k} (\Gamma^{\bar{e}e Z ,L}_{j, a} m_{e_{{j}}} (-2 \Gamma^{\bar{e}e Z ,R}_{a, c} (1 - 2 I_1) m_{e_{{a}}} + \Gamma^{\bar{e}e Z ,L}_{a, c} (1 + 2 I_2) m_{e_{{c}}}) + \Gamma^{\bar{e}e Z ,R}_{j, a} (\Gamma^{\bar{e}e Z ,R}_{a, c} (1 + 2 I_2) m^2_{e_{{j}}} - 2 \Gamma^{\bar{e}e Z ,L}_{a, c} (1 - 2 I_1) m_{e_{{a}}} m_{e_{{c}}})))/(m^2_{e_{{j}}} - m^2_{e_{{c}}}) \\ 
  OH2lSR= & ( \Gamma^{\bar{e}e h ,R}_{c, i, k} (\Gamma^{\bar{e}e Z ,R}_{j, a} m_{e_{{j}}} (-2 \Gamma^{\bar{e}e Z ,L}_{a, c} (1 - 2 I_1) m_{e_{{a}}} + \Gamma^{\bar{e}e Z ,R}_{a, c} (1 + 2 I_2) m_{e_{{c}}}) + \Gamma^{\bar{e}e Z ,L}_{j, a} (\Gamma^{\bar{e}e Z ,L}_{a, c} (1 + 2 I_2) m^2_{e_{{j}}} - 2 \Gamma^{\bar{e}e Z ,R}_{a, c} (1 - 2 I_1) m_{e_{{a}}} m_{e_{{c}}})))/(m^2_{e_{{j}}} - m^2_{e_{{c}}}) \\ 
\end{align} 


 \begin{center}
\begin{fmffile}{Diagrams/H2lWaveNumberOfConsideredExternalStatesVWm14}
\fmfframe(20,20)(20,20){
\begin{fmfgraph*}(150,75)
\fmfleft{l1,l2}
\fmfright{r1,r2}
\fmf{plain}{v3,l2}
\fmf{dashes,label=$h_{{k}}$}{v3,v4}
\fmf{phantom}{v4,r1}
\fmf{phantom}{v4,r2}
\fmf{phantom}{l1,v3}
\fmffreeze
\fmf{plain}{l1,v1}
\fmf{dashes,right,tension=0.2,label=$\tilde{\nu}_{{a}}$}{v1,v2}
\fmf{plain,left,tension=0.2,label=$\tilde{\chi}^+_{{b}}$}{v1,v2}
\fmf{plain,label=$e_{{c}}$}{v2,v3}
\fmflabel{$e_{{i}}$}{l2}
\fmflabel{$\bar{e}_{{j}}$}{l1}
\end{fmfgraph*}}
\end{fmffile}
\end{center}
 
\begin{align} 
I_1= & B_0(m^2_{e_{{j}}}, m^2_{\tilde{\chi}^-_{{b}}}, m^2_{\tilde{\nu}_{{a}}}) \\ 
I_2= & B_1(m^2_{e_{{j}}}, m^2_{\tilde{\chi}^-_{{b}}}, m^2_{\tilde{\nu}_{{a}}}) \\ 
  OH2lSL= & ( \Gamma^{\bar{e}e h ,L}_{c, i, k} (\Gamma^{\bar{e}\tilde{\chi}^- \tilde{\nu} ,L}_{j, b, a} \Gamma^{\tilde{\chi}^+e \tilde{\nu}^*,R}_{b, c, a} I_2 m^2_{e_{{j}}} - \Gamma^{\bar{e}\tilde{\chi}^- \tilde{\nu} ,R}_{j, b, a} \Gamma^{\tilde{\chi}^+e \tilde{\nu}^*,R}_{b, c, a} I_1 m_{e_{{j}}} m_{\tilde{\chi}^-_{{b}}} + \Gamma^{\bar{e}\tilde{\chi}^- \tilde{\nu} ,R}_{j, b, a} \Gamma^{\tilde{\chi}^+e \tilde{\nu}^*,L}_{b, c, a} I_2 m_{e_{{j}}} m_{e_{{c}}} - \Gamma^{\bar{e}\tilde{\chi}^- \tilde{\nu} ,L}_{j, b, a} \Gamma^{\tilde{\chi}^+e \tilde{\nu}^*,L}_{b, c, a} I_1 m_{\tilde{\chi}^-_{{b}}} m_{e_{{c}}}))/(m^2_{e_{{j}}} - m^2_{e_{{c}}}) \\ 
  OH2lSR= & ( \Gamma^{\bar{e}e h ,R}_{c, i, k} (\Gamma^{\bar{e}\tilde{\chi}^- \tilde{\nu} ,R}_{j, b, a} \Gamma^{\tilde{\chi}^+e \tilde{\nu}^*,L}_{b, c, a} I_2 m^2_{e_{{j}}} - \Gamma^{\bar{e}\tilde{\chi}^- \tilde{\nu} ,L}_{j, b, a} \Gamma^{\tilde{\chi}^+e \tilde{\nu}^*,L}_{b, c, a} I_1 m_{e_{{j}}} m_{\tilde{\chi}^-_{{b}}} + \Gamma^{\bar{e}\tilde{\chi}^- \tilde{\nu} ,L}_{j, b, a} \Gamma^{\tilde{\chi}^+e \tilde{\nu}^*,R}_{b, c, a} I_2 m_{e_{{j}}} m_{e_{{c}}} - \Gamma^{\bar{e}\tilde{\chi}^- \tilde{\nu} ,R}_{j, b, a} \Gamma^{\tilde{\chi}^+e \tilde{\nu}^*,R}_{b, c, a} I_1 m_{\tilde{\chi}^-_{{b}}} m_{e_{{c}}}))/(m^2_{e_{{j}}} - m^2_{e_{{c}}}) \\ 
\end{align} 


 \begin{center}
\begin{fmffile}{Diagrams/H2lWaveNumberOfConsideredExternalStatesVWm15}
\fmfframe(20,20)(20,20){
\begin{fmfgraph*}(150,75)
\fmfleft{l1,l2}
\fmfright{r1,r2}
\fmf{plain}{v3,l2}
\fmf{dashes,label=$h_{{k}}$}{v3,v4}
\fmf{phantom}{v4,r1}
\fmf{phantom}{v4,r2}
\fmf{phantom}{l1,v3}
\fmffreeze
\fmf{plain}{l1,v1}
\fmf{dashes,right,tension=0.2,label=$H^-_{{a}}$}{v1,v2}
\fmf{plain,left,tension=0.2,label=$\bar{\nu}_{{b}}$}{v1,v2}
\fmf{plain,label=$e_{{c}}$}{v2,v3}
\fmflabel{$e_{{i}}$}{l2}
\fmflabel{$\bar{e}_{{j}}$}{l1}
\end{fmfgraph*}}
\end{fmffile}
\end{center}
 
\begin{align} 
I_1= & B_0(m^2_{e_{{j}}}, m^2_{\nu_{{b}}}, m^2_{H^-_{{a}}}) \\ 
I_2= & B_1(m^2_{e_{{j}}}, m^2_{\nu_{{b}}}, m^2_{H^-_{{a}}}) \\ 
  OH2lSL= & ( \Gamma^{\bar{e}e h ,L}_{c, i, k} (\Gamma^{\bar{e}\nu H^- ,L}_{j, b, a} \Gamma^{\bar{\nu}e H^+,R}_{b, c, a} I_2 m^2_{e_{{j}}} - \Gamma^{\bar{e}\nu H^- ,R}_{j, b, a} \Gamma^{\bar{\nu}e H^+,R}_{b, c, a} I_1 m_{e_{{j}}} m_{\nu_{{b}}} + \Gamma^{\bar{e}\nu H^- ,R}_{j, b, a} \Gamma^{\bar{\nu}e H^+,L}_{b, c, a} I_2 m_{e_{{j}}} m_{e_{{c}}} - \Gamma^{\bar{e}\nu H^- ,L}_{j, b, a} \Gamma^{\bar{\nu}e H^+,L}_{b, c, a} I_1 m_{\nu_{{b}}} m_{e_{{c}}}))/(m^2_{e_{{j}}} - m^2_{e_{{c}}}) \\ 
  OH2lSR= & ( \Gamma^{\bar{e}e h ,R}_{c, i, k} (\Gamma^{\bar{e}\nu H^- ,R}_{j, b, a} \Gamma^{\bar{\nu}e H^+,L}_{b, c, a} I_2 m^2_{e_{{j}}} - \Gamma^{\bar{e}\nu H^- ,L}_{j, b, a} \Gamma^{\bar{\nu}e H^+,L}_{b, c, a} I_1 m_{e_{{j}}} m_{\nu_{{b}}} + \Gamma^{\bar{e}\nu H^- ,L}_{j, b, a} \Gamma^{\bar{\nu}e H^+,R}_{b, c, a} I_2 m_{e_{{j}}} m_{e_{{c}}} - \Gamma^{\bar{e}\nu H^- ,R}_{j, b, a} \Gamma^{\bar{\nu}e H^+,R}_{b, c, a} I_1 m_{\nu_{{b}}} m_{e_{{c}}}))/(m^2_{e_{{j}}} - m^2_{e_{{c}}}) \\ 
\end{align} 


 \begin{center}
\begin{fmffile}{Diagrams/H2lWaveNumberOfConsideredExternalStatesVWm16}
\fmfframe(20,20)(20,20){
\begin{fmfgraph*}(150,75)
\fmfleft{l1,l2}
\fmfright{r1,r2}
\fmf{plain}{v3,l2}
\fmf{dashes,label=$h_{{k}}$}{v3,v4}
\fmf{phantom}{v4,r1}
\fmf{phantom}{v4,r2}
\fmf{phantom}{l1,v3}
\fmffreeze
\fmf{plain}{l1,v1}
\fmf{wiggly,right,tension=0.2,label=$W^-$}{v1,v2}
\fmf{plain,left,tension=0.2,label=$\bar{\nu}_{{b}}$}{v1,v2}
\fmf{plain,label=$e_{{c}}$}{v2,v3}
\fmflabel{$e_{{i}}$}{l2}
\fmflabel{$\bar{e}_{{j}}$}{l1}
\end{fmfgraph*}}
\end{fmffile}
\end{center}
 
\begin{align} 
I_1= & B_0(m^2_{e_{{j}}}, m^2_{\nu_{{b}}}, m^2_{W^-}) \\ 
I_2= & B_1(m^2_{e_{{j}}}, m^2_{\nu_{{b}}}, m^2_{W^-}) \\ 
  OH2lSL= & ( \Gamma^{\bar{e}e h ,L}_{c, i, k} (\Gamma^{\bar{e}\nu W^- ,L}_{j, b} m_{e_{{j}}} (-2 \Gamma^{\bar{\nu}e W^+,R}_{b, c} (1 - 2 I_1) m_{\nu_{{b}}} + \Gamma^{\bar{\nu}e W^+,L}_{b, c} (1 + 2 I_2) m_{e_{{c}}}) + \Gamma^{\bar{e}\nu W^- ,R}_{j, b} (\Gamma^{\bar{\nu}e W^+,R}_{b, c} (1 + 2 I_2) m^2_{e_{{j}}} - 2 \Gamma^{\bar{\nu}e W^+,L}_{b, c} (1 - 2 I_1) m_{\nu_{{b}}} m_{e_{{c}}})))/(m^2_{e_{{j}}} - m^2_{e_{{c}}}) \\ 
  OH2lSR= & ( \Gamma^{\bar{e}e h ,R}_{c, i, k} (\Gamma^{\bar{e}\nu W^- ,R}_{j, b} m_{e_{{j}}} (-2 \Gamma^{\bar{\nu}e W^+,L}_{b, c} (1 - 2 I_1) m_{\nu_{{b}}} + \Gamma^{\bar{\nu}e W^+,R}_{b, c} (1 + 2 I_2) m_{e_{{c}}}) + \Gamma^{\bar{e}\nu W^- ,L}_{j, b} (\Gamma^{\bar{\nu}e W^+,L}_{b, c} (1 + 2 I_2) m^2_{e_{{j}}} - 2 \Gamma^{\bar{\nu}e W^+,R}_{b, c} (1 - 2 I_1) m_{\nu_{{b}}} m_{e_{{c}}})))/(m^2_{e_{{j}}} - m^2_{e_{{c}}}) \\ 
\end{align} 
\subsection{Penguin contributions, Propagator: $\text{Propagator}$} 



 \begin{center}
\begin{fmffile}{Diagrams/H2lPenguinNumberOfConsideredExternalStatesVWm1}
\fmfframe(20,20)(20,20){
\begin{fmfgraph*}(150,75)
\fmfleft{l1,l2}
\fmfright{r1,r2}
\fmf{phantom}{r1,v1}
\fmf{phantom}{v1,r2}
\fmf{plain}{l1,v2}
\fmf{plain,label=$e_{{c}}$,tension=0.3}{v2,v3}
\fmf{plain,label=$e_{{b}}$,tension=0.3}{v3,v4}
\fmf{plain}{v4,l2}
\fmf{dashes,tension=1.0,label=$h_{{k}}$}{v1,v3}
\fmf{dashes,tension=0.1,label=$A^0_{{a}}$}{v2,v4}
\fmflabel{$e_{{i}}$}{l2}
\fmflabel{$\bar{e}_{{j}}$}{l1}
\end{fmfgraph*}}
\end{fmffile}
\end{center}
 
\begin{align} 
I_1= & B_0(m^2_{h_{{k}}}, m^2_{e_{{b}}}, m^2_{e_{{c}}}) \\ 
I_2= & C_0(m^2_{h_{{k}}}, 0, 0, m^2_{e_{{c}}}, m^2_{e_{{b}}}, m^2_{A^0_{{a}}}) \\ 
  OH2lSL= & -( \Gamma^{\bar{e}e A^0 ,L}_{b, i, a} \Gamma^{\bar{e}e A^0 ,L}_{j, c, a} (\Gamma^{\bar{e}e h ,L}_{c, b, k} I_2 m_{e_{{b}}} m_{e_{{c}}} + \Gamma^{\bar{e}e h ,R}_{c, b, k} (I_1 + I_2 m^2_{A^0_{{a}}}))) \\ 
  OH2lSR= & -( \Gamma^{\bar{e}e A^0 ,R}_{b, i, a} \Gamma^{\bar{e}e A^0 ,R}_{j, c, a} (\Gamma^{\bar{e}e h ,R}_{c, b, k} I_2 m_{e_{{b}}} m_{e_{{c}}} + \Gamma^{\bar{e}e h ,L}_{c, b, k} (I_1 + I_2 m^2_{A^0_{{a}}}))) \\ 
\end{align} 


 \begin{center}
\begin{fmffile}{Diagrams/H2lPenguinNumberOfConsideredExternalStatesVWm2}
\fmfframe(20,20)(20,20){
\begin{fmfgraph*}(150,75)
\fmfleft{l1,l2}
\fmfright{r1,r2}
\fmf{phantom}{r1,v1}
\fmf{phantom}{v1,r2}
\fmf{plain}{l1,v2}
\fmf{dashes,label=$\tilde{e}_{{c}}$,tension=0.3}{v2,v3}
\fmf{dashes,label=$\tilde{e}_{{b}}$,tension=0.3}{v3,v4}
\fmf{plain}{v4,l2}
\fmf{dashes,tension=1.0,label=$h_{{k}}$}{v1,v3}
\fmf{plain,tension=0.1,label=$\tilde{\chi}^0_{{a}}$}{v2,v4}
\fmflabel{$e_{{i}}$}{l2}
\fmflabel{$\bar{e}_{{j}}$}{l1}
\end{fmfgraph*}}
\end{fmffile}
\end{center}
 
\begin{align} 
I_1= & C_0(0, m^2_{h_{{k}}}, 0, m^2_{\tilde{\chi}^0_{{a}}}, m^2_{\tilde{e}_{{c}}}, m^2_{\tilde{e}_{{b}}}) \\ 
  OH2lSL= & -( \Gamma^{\tilde{\chi}^0 e \tilde{e}^*,L}_{a, i, b} \Gamma^{\bar{e}\tilde{\chi}^0 \tilde{e} ,L}_{j, a, c} \Gamma^{h \tilde{e} \tilde{e}^*}_{k, b, c} I_1 m_{\tilde{\chi}^0_{{a}}}) \\ 
  OH2lSR= & -( \Gamma^{\tilde{\chi}^0 e \tilde{e}^*,R}_{a, i, b} \Gamma^{\bar{e}\tilde{\chi}^0 \tilde{e} ,R}_{j, a, c} \Gamma^{h \tilde{e} \tilde{e}^*}_{k, b, c} I_1 m_{\tilde{\chi}^0_{{a}}}) \\ 
\end{align} 


 \begin{center}
\begin{fmffile}{Diagrams/H2lPenguinNumberOfConsideredExternalStatesVWm3}
\fmfframe(20,20)(20,20){
\begin{fmfgraph*}(150,75)
\fmfleft{l1,l2}
\fmfright{r1,r2}
\fmf{phantom}{r1,v1}
\fmf{phantom}{v1,r2}
\fmf{plain}{l1,v2}
\fmf{plain,label=$e_{{c}}$,tension=0.3}{v2,v3}
\fmf{plain,label=$e_{{b}}$,tension=0.3}{v3,v4}
\fmf{plain}{v4,l2}
\fmf{dashes,tension=1.0,label=$h_{{k}}$}{v1,v3}
\fmf{dashes,tension=0.1,label=$h_{{a}}$}{v2,v4}
\fmflabel{$e_{{i}}$}{l2}
\fmflabel{$\bar{e}_{{j}}$}{l1}
\end{fmfgraph*}}
\end{fmffile}
\end{center}
 
\begin{align} 
I_1= & B_0(m^2_{h_{{k}}}, m^2_{e_{{b}}}, m^2_{e_{{c}}}) \\ 
I_2= & C_0(m^2_{h_{{k}}}, 0, 0, m^2_{e_{{c}}}, m^2_{e_{{b}}}, m^2_{h_{{a}}}) \\ 
  OH2lSL= & -( \Gamma^{\bar{e}e h ,L}_{b, i, a} \Gamma^{\bar{e}e h ,L}_{j, c, a} (\Gamma^{\bar{e}e h ,L}_{c, b, k} I_2 m_{e_{{b}}} m_{e_{{c}}} + \Gamma^{\bar{e}e h ,R}_{c, b, k} (I_1 + I_2 m^2_{h_{{a}}}))) \\ 
  OH2lSR= & -( \Gamma^{\bar{e}e h ,R}_{b, i, a} \Gamma^{\bar{e}e h ,R}_{j, c, a} (\Gamma^{\bar{e}e h ,R}_{c, b, k} I_2 m_{e_{{b}}} m_{e_{{c}}} + \Gamma^{\bar{e}e h ,L}_{c, b, k} (I_1 + I_2 m^2_{h_{{a}}}))) \\ 
\end{align} 


 \begin{center}
\begin{fmffile}{Diagrams/H2lPenguinNumberOfConsideredExternalStatesVWm4}
\fmfframe(20,20)(20,20){
\begin{fmfgraph*}(150,75)
\fmfleft{l1,l2}
\fmfright{r1,r2}
\fmf{phantom}{r1,v1}
\fmf{phantom}{v1,r2}
\fmf{plain}{l1,v2}
\fmf{plain,label=$e_{{c}}$,tension=0.3}{v2,v3}
\fmf{plain,label=$e_{{b}}$,tension=0.3}{v3,v4}
\fmf{plain}{v4,l2}
\fmf{dashes,tension=1.0,label=$h_{{k}}$}{v1,v3}
\fmf{wiggly,tension=0.1,label=$Z$}{v2,v4}
\fmflabel{$e_{{i}}$}{l2}
\fmflabel{$\bar{e}_{{j}}$}{l1}
\end{fmfgraph*}}
\end{fmffile}
\end{center}
 
\begin{align} 
I_1= & B_0(m^2_{h_{{k}}}, m^2_{e_{{b}}}, m^2_{e_{{c}}}) \\ 
I_2= & C_0(m^2_{h_{{k}}}, 0, 0, m^2_{e_{{c}}}, m^2_{e_{{b}}}, m^2_{Z}) \\ 
  OH2lSL= & 2  \Gamma^{\bar{e}e Z ,L}_{b, i} \Gamma^{\bar{e}e Z ,R}_{j, c} (2 \Gamma^{\bar{e}e h ,R}_{c, b, k} I_2 m_{e_{{b}}} m_{e_{{c}}} + \Gamma^{\bar{e}e h ,L}_{c, b, k} (-1 + 2 (I_1 + I_2 m^2_{Z}))) \\ 
  OH2lSR= & 2  \Gamma^{\bar{e}e Z ,R}_{b, i} \Gamma^{\bar{e}e Z ,L}_{j, c} (2 \Gamma^{\bar{e}e h ,L}_{c, b, k} I_2 m_{e_{{b}}} m_{e_{{c}}} + \Gamma^{\bar{e}e h ,R}_{c, b, k} (-1 + 2 (I_1 + I_2 m^2_{Z}))) \\ 
\end{align} 


 \begin{center}
\begin{fmffile}{Diagrams/H2lPenguinNumberOfConsideredExternalStatesVWm5}
\fmfframe(20,20)(20,20){
\begin{fmfgraph*}(150,75)
\fmfleft{l1,l2}
\fmfright{r1,r2}
\fmf{phantom}{r1,v1}
\fmf{phantom}{v1,r2}
\fmf{plain}{l1,v2}
\fmf{dashes,label=$\tilde{\nu}_{{c}}$,tension=0.3}{v2,v3}
\fmf{dashes,label=$\tilde{\nu}_{{b}}$,tension=0.3}{v3,v4}
\fmf{plain}{v4,l2}
\fmf{dashes,tension=1.0,label=$h_{{k}}$}{v1,v3}
\fmf{plain,tension=0.1,label=$\tilde{\chi}^+_{{a}}$}{v2,v4}
\fmflabel{$e_{{i}}$}{l2}
\fmflabel{$\bar{e}_{{j}}$}{l1}
\end{fmfgraph*}}
\end{fmffile}
\end{center}
 
\begin{align} 
I_1= & C_0(0, m^2_{h_{{k}}}, 0, m^2_{\tilde{\chi}^-_{{a}}}, m^2_{\tilde{\nu}_{{c}}}, m^2_{\tilde{\nu}_{{b}}}) \\ 
  OH2lSL= & -( \Gamma^{\tilde{\chi}^+e \tilde{\nu}^*,L}_{a, i, b} \Gamma^{\bar{e}\tilde{\chi}^- \tilde{\nu} ,L}_{j, a, c} \Gamma^{h \tilde{\nu} \tilde{\nu}^*}_{k, b, c} I_1 m_{\tilde{\chi}^-_{{a}}}) \\ 
  OH2lSR= & -( \Gamma^{\tilde{\chi}^+e \tilde{\nu}^*,R}_{a, i, b} \Gamma^{\bar{e}\tilde{\chi}^- \tilde{\nu} ,R}_{j, a, c} \Gamma^{h \tilde{\nu} \tilde{\nu}^*}_{k, b, c} I_1 m_{\tilde{\chi}^-_{{a}}}) \\ 
\end{align} 


 \begin{center}
\begin{fmffile}{Diagrams/H2lPenguinNumberOfConsideredExternalStatesVWm6}
\fmfframe(20,20)(20,20){
\begin{fmfgraph*}(150,75)
\fmfleft{l1,l2}
\fmfright{r1,r2}
\fmf{phantom}{r1,v1}
\fmf{phantom}{v1,r2}
\fmf{plain}{l1,v2}
\fmf{dashes,label=$A^0_{{c}}$,tension=0.3}{v2,v3}
\fmf{dashes,label=$A^0_{{b}}$,tension=0.3}{v3,v4}
\fmf{plain}{v4,l2}
\fmf{dashes,tension=1.0,label=$h_{{k}}$}{v1,v3}
\fmf{plain,tension=0.1,label=$\bar{e}_{{a}}$}{v2,v4}
\fmflabel{$e_{{i}}$}{l2}
\fmflabel{$\bar{e}_{{j}}$}{l1}
\end{fmfgraph*}}
\end{fmffile}
\end{center}
 
\begin{align} 
I_1= & C_0(0, m^2_{h_{{k}}}, 0, m^2_{e_{{a}}}, m^2_{A^0_{{c}}}, m^2_{A^0_{{b}}}) \\ 
  OH2lSL= & -( \Gamma^{\bar{e}e A^0 ,L}_{a, i, b} \Gamma^{\bar{e}e A^0 ,L}_{j, a, c} \Gamma^{A^0 A^0 h }_{c, b, k} I_1 m_{e_{{a}}}) \\ 
  OH2lSR= & -( \Gamma^{\bar{e}e A^0 ,R}_{a, i, b} \Gamma^{\bar{e}e A^0 ,R}_{j, a, c} \Gamma^{A^0 A^0 h }_{c, b, k} I_1 m_{e_{{a}}}) \\ 
\end{align} 


 \begin{center}
\begin{fmffile}{Diagrams/H2lPenguinNumberOfConsideredExternalStatesVWm7}
\fmfframe(20,20)(20,20){
\begin{fmfgraph*}(150,75)
\fmfleft{l1,l2}
\fmfright{r1,r2}
\fmf{phantom}{r1,v1}
\fmf{phantom}{v1,r2}
\fmf{plain}{l1,v2}
\fmf{dashes,label=$A^0_{{c}}$,tension=0.3}{v2,v3}
\fmf{wiggly,label=$Z$,tension=0.3}{v3,v4}
\fmf{plain}{v4,l2}
\fmf{dashes,tension=1.0,label=$h_{{k}}$}{v1,v3}
\fmf{plain,tension=0.1,label=$\bar{e}_{{a}}$}{v2,v4}
\fmflabel{$e_{{i}}$}{l2}
\fmflabel{$\bar{e}_{{j}}$}{l1}
\end{fmfgraph*}}
\end{fmffile}
\end{center}
 
\begin{align} 
I_1= & B_0(m^2_{h_{{k}}}, m^2_{A^0_{{c}}}, m^2_{Z}) \\ 
I_2= & C_0(0, m^2_{h_{{k}}}, 0, m^2_{e_{{a}}}, m^2_{A^0_{{c}}}, m^2_{Z}) \\ 
  OH2lSL= & -( \Gamma^{\bar{e}e Z ,L}_{a, i} \Gamma^{\bar{e}e A^0 ,L}_{j, a, c} (- \Gamma^{A^0 h Z } _{c, k}) (I_1 + I_2 m^2_{e_{{a}}})) \\ 
  OH2lSR= & -( \Gamma^{\bar{e}e Z ,R}_{a, i} \Gamma^{\bar{e}e A^0 ,R}_{j, a, c} (- \Gamma^{A^0 h Z } _{c, k}) (I_1 + I_2 m^2_{e_{{a}}})) \\ 
\end{align} 


 \begin{center}
\begin{fmffile}{Diagrams/H2lPenguinNumberOfConsideredExternalStatesVWm8}
\fmfframe(20,20)(20,20){
\begin{fmfgraph*}(150,75)
\fmfleft{l1,l2}
\fmfright{r1,r2}
\fmf{phantom}{r1,v1}
\fmf{phantom}{v1,r2}
\fmf{plain}{l1,v2}
\fmf{dashes,label=$h_{{c}}$,tension=0.3}{v2,v3}
\fmf{dashes,label=$h_{{b}}$,tension=0.3}{v3,v4}
\fmf{plain}{v4,l2}
\fmf{dashes,tension=1.0,label=$h_{{k}}$}{v1,v3}
\fmf{plain,tension=0.1,label=$\bar{e}_{{a}}$}{v2,v4}
\fmflabel{$e_{{i}}$}{l2}
\fmflabel{$\bar{e}_{{j}}$}{l1}
\end{fmfgraph*}}
\end{fmffile}
\end{center}
 
\begin{align} 
I_1= & C_0(0, m^2_{h_{{k}}}, 0, m^2_{e_{{a}}}, m^2_{h_{{c}}}, m^2_{h_{{b}}}) \\ 
  OH2lSL= & -( \Gamma^{\bar{e}e h ,L}_{a, i, b} \Gamma^{\bar{e}e h ,L}_{j, a, c} \Gamma^{h h h }_{k, c, b} I_1 m_{e_{{a}}}) \\ 
  OH2lSR= & -( \Gamma^{\bar{e}e h ,R}_{a, i, b} \Gamma^{\bar{e}e h ,R}_{j, a, c} \Gamma^{h h h }_{k, c, b} I_1 m_{e_{{a}}}) \\ 
\end{align} 


 \begin{center}
\begin{fmffile}{Diagrams/H2lPenguinNumberOfConsideredExternalStatesVWm9}
\fmfframe(20,20)(20,20){
\begin{fmfgraph*}(150,75)
\fmfleft{l1,l2}
\fmfright{r1,r2}
\fmf{phantom}{r1,v1}
\fmf{phantom}{v1,r2}
\fmf{plain}{l1,v2}
\fmf{wiggly,label=$Z$,tension=0.3}{v2,v3}
\fmf{dashes,label=$A^0_{{b}}$,tension=0.3}{v3,v4}
\fmf{plain}{v4,l2}
\fmf{dashes,tension=1.0,label=$h_{{k}}$}{v1,v3}
\fmf{plain,tension=0.1,label=$\bar{e}_{{a}}$}{v2,v4}
\fmflabel{$e_{{i}}$}{l2}
\fmflabel{$\bar{e}_{{j}}$}{l1}
\end{fmfgraph*}}
\end{fmffile}
\end{center}
 
\begin{align} 
I_1= & B_0(m^2_{h_{{k}}}, m^2_{A^0_{{b}}}, m^2_{Z}) \\ 
I_2= & C_0(0, m^2_{h_{{k}}}, 0, m^2_{e_{{a}}}, m^2_{Z}, m^2_{A^0_{{b}}}) \\ 
  OH2lSL= &  \Gamma^{\bar{e}e A^0 ,L}_{a, i, b} \Gamma^{\bar{e}e Z ,R}_{j, a} (- \Gamma^{A^0 h Z } _{b, k}) (I_1 + I_2 m^2_{e_{{a}}}) \\ 
  OH2lSR= &  \Gamma^{\bar{e}e A^0 ,R}_{a, i, b} \Gamma^{\bar{e}e Z ,L}_{j, a} (- \Gamma^{A^0 h Z } _{b, k}) (I_1 + I_2 m^2_{e_{{a}}}) \\ 
\end{align} 


 \begin{center}
\begin{fmffile}{Diagrams/H2lPenguinNumberOfConsideredExternalStatesVWm10}
\fmfframe(20,20)(20,20){
\begin{fmfgraph*}(150,75)
\fmfleft{l1,l2}
\fmfright{r1,r2}
\fmf{phantom}{r1,v1}
\fmf{phantom}{v1,r2}
\fmf{plain}{l1,v2}
\fmf{wiggly,label=$Z$,tension=0.3}{v2,v3}
\fmf{wiggly,label=$Z$,tension=0.3}{v3,v4}
\fmf{plain}{v4,l2}
\fmf{dashes,tension=1.0,label=$h_{{k}}$}{v1,v3}
\fmf{plain,tension=0.1,label=$\bar{e}_{{a}}$}{v2,v4}
\fmflabel{$e_{{i}}$}{l2}
\fmflabel{$\bar{e}_{{j}}$}{l1}
\end{fmfgraph*}}
\end{fmffile}
\end{center}
 
\begin{align} 
I_1= & C_0(0, m^2_{h_{{k}}}, 0, m^2_{e_{{a}}}, m^2_{Z}, m^2_{Z}) \\ 
  OH2lSL= & -4  \Gamma^{\bar{e}e Z ,L}_{a, i} \Gamma^{\bar{e}e Z ,R}_{j, a} \Gamma^{h Z Z }_{k} I_1 m_{e_{{a}}} \\ 
  OH2lSR= & -4  \Gamma^{\bar{e}e Z ,R}_{a, i} \Gamma^{\bar{e}e Z ,L}_{j, a} \Gamma^{h Z Z }_{k} I_1 m_{e_{{a}}} \\ 
\end{align} 


 \begin{center}
\begin{fmffile}{Diagrams/H2lPenguinNumberOfConsideredExternalStatesVWm11}
\fmfframe(20,20)(20,20){
\begin{fmfgraph*}(150,75)
\fmfleft{l1,l2}
\fmfright{r1,r2}
\fmf{phantom}{r1,v1}
\fmf{phantom}{v1,r2}
\fmf{plain}{l1,v2}
\fmf{dashes,label=$H^-_{{c}}$,tension=0.3}{v2,v3}
\fmf{dashes,label=$H^-_{{b}}$,tension=0.3}{v3,v4}
\fmf{plain}{v4,l2}
\fmf{dashes,tension=1.0,label=$h_{{k}}$}{v1,v3}
\fmf{plain,tension=0.1,label=$\bar{\nu}_{{a}}$}{v2,v4}
\fmflabel{$e_{{i}}$}{l2}
\fmflabel{$\bar{e}_{{j}}$}{l1}
\end{fmfgraph*}}
\end{fmffile}
\end{center}
 
\begin{align} 
I_1= & C_0(0, m^2_{h_{{k}}}, 0, m^2_{\nu_{{a}}}, m^2_{H^-_{{c}}}, m^2_{H^-_{{b}}}) \\ 
  OH2lSL= & -( \Gamma^{\bar{\nu}e H^+,L}_{a, i, b} \Gamma^{\bar{e}\nu H^- ,L}_{j, a, c} \Gamma^{h H^- H^+}_{k, b, c} I_1 m_{\nu_{{a}}}) \\ 
  OH2lSR= & -( \Gamma^{\bar{\nu}e H^+,R}_{a, i, b} \Gamma^{\bar{e}\nu H^- ,R}_{j, a, c} \Gamma^{h H^- H^+}_{k, b, c} I_1 m_{\nu_{{a}}}) \\ 
\end{align} 


 \begin{center}
\begin{fmffile}{Diagrams/H2lPenguinNumberOfConsideredExternalStatesVWm12}
\fmfframe(20,20)(20,20){
\begin{fmfgraph*}(150,75)
\fmfleft{l1,l2}
\fmfright{r1,r2}
\fmf{phantom}{r1,v1}
\fmf{phantom}{v1,r2}
\fmf{plain}{l1,v2}
\fmf{dashes,label=$H^-_{{c}}$,tension=0.3}{v2,v3}
\fmf{wiggly,label=$W^-$,tension=0.3}{v3,v4}
\fmf{plain}{v4,l2}
\fmf{dashes,tension=1.0,label=$h_{{k}}$}{v1,v3}
\fmf{plain,tension=0.1,label=$\bar{\nu}_{{a}}$}{v2,v4}
\fmflabel{$e_{{i}}$}{l2}
\fmflabel{$\bar{e}_{{j}}$}{l1}
\end{fmfgraph*}}
\end{fmffile}
\end{center}
 
\begin{align} 
I_1= & B_0(m^2_{h_{{k}}}, m^2_{H^-_{{c}}}, m^2_{W^-}) \\ 
I_2= & C_0(0, m^2_{h_{{k}}}, 0, m^2_{\nu_{{a}}}, m^2_{H^-_{{c}}}, m^2_{W^-}) \\ 
  OH2lSL= & -( \Gamma^{\bar{\nu}e W^+,L}_{a, i} \Gamma^{\bar{e}\nu H^- ,L}_{j, a, c} (- \Gamma^{h H^+W^- } _{k, c}) (I_1 + I_2 m^2_{\nu_{{a}}})) \\ 
  OH2lSR= & -( \Gamma^{\bar{\nu}e W^+,R}_{a, i} \Gamma^{\bar{e}\nu H^- ,R}_{j, a, c} (- \Gamma^{h H^+W^- } _{k, c}) (I_1 + I_2 m^2_{\nu_{{a}}})) \\ 
\end{align} 


 \begin{center}
\begin{fmffile}{Diagrams/H2lPenguinNumberOfConsideredExternalStatesVWm13}
\fmfframe(20,20)(20,20){
\begin{fmfgraph*}(150,75)
\fmfleft{l1,l2}
\fmfright{r1,r2}
\fmf{phantom}{r1,v1}
\fmf{phantom}{v1,r2}
\fmf{plain}{l1,v2}
\fmf{wiggly,label=$W^-$,tension=0.3}{v2,v3}
\fmf{dashes,label=$H^-_{{b}}$,tension=0.3}{v3,v4}
\fmf{plain}{v4,l2}
\fmf{dashes,tension=1.0,label=$h_{{k}}$}{v1,v3}
\fmf{plain,tension=0.1,label=$\bar{\nu}_{{a}}$}{v2,v4}
\fmflabel{$e_{{i}}$}{l2}
\fmflabel{$\bar{e}_{{j}}$}{l1}
\end{fmfgraph*}}
\end{fmffile}
\end{center}
 
\begin{align} 
I_1= & B_0(m^2_{h_{{k}}}, m^2_{H^-_{{b}}}, m^2_{W^-}) \\ 
I_2= & C_0(0, m^2_{h_{{k}}}, 0, m^2_{\nu_{{a}}}, m^2_{W^-}, m^2_{H^-_{{b}}}) \\ 
  OH2lSL= &  \Gamma^{\bar{\nu}e H^+,L}_{a, i, b} \Gamma^{\bar{e}\nu W^- ,R}_{j, a} (- \Gamma^{h H^- W^+} _{k, b}) (I_1 + I_2 m^2_{\nu_{{a}}}) \\ 
  OH2lSR= &  \Gamma^{\bar{\nu}e H^+,R}_{a, i, b} \Gamma^{\bar{e}\nu W^- ,L}_{j, a} (- \Gamma^{h H^- W^+} _{k, b}) (I_1 + I_2 m^2_{\nu_{{a}}}) \\ 
\end{align} 


 \begin{center}
\begin{fmffile}{Diagrams/H2lPenguinNumberOfConsideredExternalStatesVWm14}
\fmfframe(20,20)(20,20){
\begin{fmfgraph*}(150,75)
\fmfleft{l1,l2}
\fmfright{r1,r2}
\fmf{phantom}{r1,v1}
\fmf{phantom}{v1,r2}
\fmf{plain}{l1,v2}
\fmf{wiggly,label=$W^-$,tension=0.3}{v2,v3}
\fmf{wiggly,label=$W^-$,tension=0.3}{v3,v4}
\fmf{plain}{v4,l2}
\fmf{dashes,tension=1.0,label=$h_{{k}}$}{v1,v3}
\fmf{plain,tension=0.1,label=$\bar{\nu}_{{a}}$}{v2,v4}
\fmflabel{$e_{{i}}$}{l2}
\fmflabel{$\bar{e}_{{j}}$}{l1}
\end{fmfgraph*}}
\end{fmffile}
\end{center}
 
\begin{align} 
I_1= & C_0(0, m^2_{h_{{k}}}, 0, m^2_{\nu_{{a}}}, m^2_{W^-}, m^2_{W^-}) \\ 
  OH2lSL= & -4  \Gamma^{\bar{\nu}e W^+,L}_{a, i} \Gamma^{\bar{e}\nu W^- ,R}_{j, a} \Gamma^{h W^+W^- }_{k} I_1 m_{\nu_{{a}}} \\ 
  OH2lSR= & -4  \Gamma^{\bar{\nu}e W^+,R}_{a, i} \Gamma^{\bar{e}\nu W^- ,L}_{j, a} \Gamma^{h W^+W^- }_{k} I_1 m_{\nu_{{a}}} \\ 
\end{align} 


 \begin{center}
\begin{fmffile}{Diagrams/H2lPenguinNumberOfConsideredExternalStatesVWm15}
\fmfframe(20,20)(20,20){
\begin{fmfgraph*}(150,75)
\fmfleft{l1,l2}
\fmfright{r1,r2}
\fmf{phantom}{r1,v1}
\fmf{phantom}{v1,r2}
\fmf{plain}{l1,v2}
\fmf{plain,label=$\tilde{\chi}^0_{{c}}$,tension=0.3}{v2,v3}
\fmf{plain,label=$\tilde{\chi}^0_{{b}}$,tension=0.3}{v3,v4}
\fmf{plain}{v4,l2}
\fmf{dashes,tension=1.0,label=$h_{{k}}$}{v1,v3}
\fmf{dashes,tension=0.1,label=$\tilde{e}^*_{{a}}$}{v2,v4}
\fmflabel{$e_{{i}}$}{l2}
\fmflabel{$\bar{e}_{{j}}$}{l1}
\end{fmfgraph*}}
\end{fmffile}
\end{center}
 
\begin{align} 
I_1= & B_0(m^2_{h_{{k}}}, m^2_{\tilde{\chi}^0_{{b}}}, m^2_{\tilde{\chi}^0_{{c}}}) \\ 
I_2= & C_0(m^2_{h_{{k}}}, 0, 0, m^2_{\tilde{\chi}^0_{{c}}}, m^2_{\tilde{\chi}^0_{{b}}}, m^2_{\tilde{e}_{{a}}}) \\ 
  OH2lSL= & -( \Gamma^{\tilde{\chi}^0 e \tilde{e}^*,L}_{b, i, a} \Gamma^{\bar{e}\tilde{\chi}^0 \tilde{e} ,L}_{j, c, a} (\Gamma^{\tilde{\chi}^0 \tilde{\chi}^0 h ,L}_{c, b, k} I_2 m_{\tilde{\chi}^0_{{b}}} m_{\tilde{\chi}^0_{{c}}} + \Gamma^{\tilde{\chi}^0 \tilde{\chi}^0 h ,R}_{c, b, k} (I_1 + I_2 m^2_{\tilde{e}_{{a}}}))) \\ 
  OH2lSR= & -( \Gamma^{\tilde{\chi}^0 e \tilde{e}^*,R}_{b, i, a} \Gamma^{\bar{e}\tilde{\chi}^0 \tilde{e} ,R}_{j, c, a} (\Gamma^{\tilde{\chi}^0 \tilde{\chi}^0 h ,R}_{c, b, k} I_2 m_{\tilde{\chi}^0_{{b}}} m_{\tilde{\chi}^0_{{c}}} + \Gamma^{\tilde{\chi}^0 \tilde{\chi}^0 h ,L}_{c, b, k} (I_1 + I_2 m^2_{\tilde{e}_{{a}}}))) \\ 
\end{align} 


 \begin{center}
\begin{fmffile}{Diagrams/H2lPenguinNumberOfConsideredExternalStatesVWm16}
\fmfframe(20,20)(20,20){
\begin{fmfgraph*}(150,75)
\fmfleft{l1,l2}
\fmfright{r1,r2}
\fmf{phantom}{r1,v1}
\fmf{phantom}{v1,r2}
\fmf{plain}{l1,v2}
\fmf{plain,label=$\tilde{\chi}^-_{{c}}$,tension=0.3}{v2,v3}
\fmf{plain,label=$\tilde{\chi}^-_{{b}}$,tension=0.3}{v3,v4}
\fmf{plain}{v4,l2}
\fmf{dashes,tension=1.0,label=$h_{{k}}$}{v1,v3}
\fmf{dashes,tension=0.1,label=$\tilde{\nu}^*_{{a}}$}{v2,v4}
\fmflabel{$e_{{i}}$}{l2}
\fmflabel{$\bar{e}_{{j}}$}{l1}
\end{fmfgraph*}}
\end{fmffile}
\end{center}
 
\begin{align} 
I_1= & B_0(m^2_{h_{{k}}}, m^2_{\tilde{\chi}^-_{{b}}}, m^2_{\tilde{\chi}^-_{{c}}}) \\ 
I_2= & C_0(m^2_{h_{{k}}}, 0, 0, m^2_{\tilde{\chi}^-_{{c}}}, m^2_{\tilde{\chi}^-_{{b}}}, m^2_{\tilde{\nu}_{{a}}}) \\ 
  OH2lSL= & -( \Gamma^{\tilde{\chi}^+e \tilde{\nu}^*,L}_{b, i, a} \Gamma^{\bar{e}\tilde{\chi}^- \tilde{\nu} ,L}_{j, c, a} (\Gamma^{\tilde{\chi}^+\tilde{\chi}^- h ,L}_{c, b, k} I_2 m_{\tilde{\chi}^-_{{b}}} m_{\tilde{\chi}^-_{{c}}} + \Gamma^{\tilde{\chi}^+\tilde{\chi}^- h ,R}_{c, b, k} (I_1 + I_2 m^2_{\tilde{\nu}_{{a}}}))) \\ 
  OH2lSR= & -( \Gamma^{\tilde{\chi}^+e \tilde{\nu}^*,R}_{b, i, a} \Gamma^{\bar{e}\tilde{\chi}^- \tilde{\nu} ,R}_{j, c, a} (\Gamma^{\tilde{\chi}^+\tilde{\chi}^- h ,R}_{c, b, k} I_2 m_{\tilde{\chi}^-_{{b}}} m_{\tilde{\chi}^-_{{c}}} + \Gamma^{\tilde{\chi}^+\tilde{\chi}^- h ,L}_{c, b, k} (I_1 + I_2 m^2_{\tilde{\nu}_{{a}}}))) \\ 
\end{align} 
\end{document}
